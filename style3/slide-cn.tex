\documentclass{kw}

% 自定义标题页内容
\title[副标题]{知识工场PPT模板测试\\样例三}
\subtitle{副标题}
\author{作者 1\inst{1}, 作者 2\inst{1}, and 作者 3\inst{2}}
\institute[]{%
	\inst{1} 复旦大学知识工场实验室\\
	\inst{2} 复旦大学计算机科学与技术学院
}
\date{\today} % 当前日期

% 开始文档
\begin{document}
\begin{frame}[plain]
	\titlepage
\end{frame}

\begin{frame}{目录}
	\tableofcontents % 这将显示整个文档的大纲
\end{frame}

\section{动机}
\begin{frame}{动机}
	\begin{itemize}
		\item 这里增加相关的段落
	\end{itemize}
	
	\seprule
	
	这里是用横线分开的内容,例如变量解释等。
\end{frame}
\begin{frame}
	当然,如果你不需要标题的话,也是可以的,例如当前这样的一个页面
	\begin{itemize}
		\item 当前这个页面并不需要标题。
	\end{itemize}
\end{frame}

\section{数学表达式}
\begin{frame}{积分和其他的表达形式}
	\begin{equation}
		\iint_{\partial\Omega}f(x)\diff{x} \in \complexes
	\end{equation}
	\begin{align}
		E &= mc^2\\
		F &= ma
	\end{align}

	\seprule
	
	\begin{tabular}{rl}
		$m$ & 质量的大小\\
		$c$ & 光速的大小
	\end{tabular}
\end{frame}
\begin{frame}{定理,公式等表示方式}
	\begin{theorem}
		下面的叙述是正确的。
		\begin{equation}
			\frac{\partial f(\vec{x})}{\partial x_i} = \sum_{l=1}^{L}\cos\left(l\frac{2\pi}{L} + 0\right)
		\end{equation}
	\end{theorem}
\end{frame}

\section{相关元素}

\begin{frame}[fragile]{Typography}
\begin{verbatim}当前的主题提供了默认值为
\emph{文字强调}文本,\alert{文字提醒}部分
或显示\textbf{文字加粗}结果。\end{verbatim}
	
\begin{center}变成下面的形式\end{center}
	
当前的主题提供了默认值为\emph{文字强调},\alert{文字提醒}部分
或显示\textbf{文字加粗}结果。

\end{frame}



\begin{frame}{字体特征测试}
	\begin{itemize}
		\item Regular
		\item \textit{Italic}
		\item \textsc{Small Caps}
		\item \textbf{Bold}
		\item \textbf{\textit{Bold Italic}}
		\item \textbf{\textsc{Bold Small Caps}}
		\item \texttt{Monospace}
		\item \texttt{\textit{Monospace Italic}}
		\item \texttt{\textbf{Monospace Bold}}
		\item \texttt{\textbf{\textit{Monospace Bold Italic}}}
	\end{itemize}
\end{frame}

\begin{frame}{列表表示}
	\begin{columns}[T,onlytextwidth]
		\column{0.33\textwidth}
		Items
		\begin{itemize}
			\item 牛奶 \item 鸡蛋 \item 土豆
		\end{itemize}
		
		\column{0.33\textwidth}
		Enumerations
		\begin{enumerate}
			\item 第一,\item 第二 和 \item 最后。
		\end{enumerate}
		
		\column{0.33\textwidth}
		Descriptions
		\begin{description}
			\item[PowerPoint] Meeh. \item[Beamer] Yeeeha.
		\end{description}
	\end{columns}
\end{frame}
\begin{frame}{表格的画法}
	\begin{table}
		\caption{世界上最大的城市(来源:维基百科)}
		\begin{tabular}{@{} lr @{}}
			\toprule
			City & Population\\
			\midrule
			Mexico City & 20,116,842\\
			Shanghai & 19,210,000\\
			Peking & 15,796,450\\
			Istanbul & 14,160,467\\
			\bottomrule
		\end{tabular}
	\end{table}
\end{frame}
\begin{frame}{模块}
	预定义了三种不同的块环境,可以用不同的可选背景颜色。
	
	\begin{columns}[T,onlytextwidth]
		\column{0.45\textwidth}
		\begin{block}{默认的}
			模块文本
		\end{block}
		
		\begin{alertblock}{提醒}
			模块文本
		\end{alertblock}
		
		\begin{exampleblock}{例子}
			模块文本
		\end{exampleblock}
		
		\column{0.45\textwidth}
		
		\begin{block}{默认}
			模块文本
		\end{block}
		
		\begin{alertblock}{提醒}
			模块文本
		\end{alertblock}
		
		\begin{exampleblock}{例子}
			模块文本
		\end{exampleblock}
		
	\end{columns}
\end{frame}
\begin{frame}{图的画法}
可以使用tikz和pdgplots作非常复杂的图像,例如下面的简单图像
	\begin{figure}
		\begin{tikzpicture}
			\begin{axis}[
				betterplot,
				width=0.9\textwidth,
				height=6cm,
				]
				
				\addplot {sin(deg(x))};
				\addplot+[samples=100] {sin(deg(2*x))};
				
			\end{axis}
		\end{tikzpicture}
	\end{figure}
\end{frame}

\begin{frame}{结束}
	\begin{center}
		\Huge 谢谢观看!
	\end{center}
\end{frame}


\appendix
\begin{frame}[fragile]{FAQs}

有时,在演示文稿末尾添加幻灯片以供观众提问时参考是有用的。
	
最好的方法是在前言中包含\verb|appendixnumberbimer|包,并在备份幻灯片之前调用\verb|| \appendix|。
	
该主题将自动关闭附录中幻灯片的幻灯片编号和进度条。

\end{frame}

\end{document}
